
\documentclass[journal,12pt,twocolumn]{IEEEtran}
\usepackage{graphicx}
\graphicspath{{./figs/}}{}
\usepackage{amsmath,amssymb,amsfonts}
\usepackage{gensymb}
\usepackage{stackengine}
\usepackage{scalerel}
\newcommand{\myvec}[1]{\ensuremath{\begin{pmatrix}#1\end{pmatrix}}}

\let\vec\mathbf

\newlength\triwidth
\newcommand\tridelt[1]{%
  \setlength\triwidth{\widthof{#1\ }}%
  \stackengine{-.1\triwidth}{#1\ }%
    {\scaleto{\Delta}{1\triwidth}}{O}{c}{F}{F}{L}%
}
\title{
Matrix-circle
}
\author{Kukunuri Sampath Govardhan}
\begin{document}
\maketitle
\tableofcontents
\section{Problem Statement}
\begin{flushleft}
    If circles x$^2$+y$^2$+2x+2ky+6 = 0,  x$^2$+y$^2$+2ky+k = 0\\
intersect orthogonally then find k.\\
\end{flushleft}
\section{Construction}
\begin{figure}[h]
    \centering
\includegraphics[width=\columnwidth]{figs/Assignment5.png}
    \caption{Orthogonal Circles}
    \label{fig:my_label}
\end{figure}

\begin{table}[h]
    \centering
    \begin{tabular}{|c|c|c|}
       \hline
       \textbf{Symbol}&\textbf{Value}&\textbf{Description}  \\
       \hline
        $\vec{C1}$ & $\begin{pmatrix}
  -1\\
  -k\\
 \end{pmatrix}$% 
 & Center of circle C1\\
        \hline
        $\vec{C2}$ & $\begin{pmatrix}
  0\\
  -k\\
 \end{pmatrix}$% 
 & Center of circle C2\\
        \hline
        $\vec{P}$ &  $\vec{X}$ & Radius of circle C1 \\
        \hline
        $\theta$ & 90\textdegree & Given that C1 and C2 are Orthogonal\\
        \hline
        d & 1 & Distance between centers of the circles\\
        \hline
    \end{tabular}
    \caption{Parameters}
    \label{tab:my_label}
\end{table}
\vspace{3cm}
\begin{figure}[h]
    \centering
\includegraphics[width=\columnwidth]{figs/Assignment5py.png}
    \caption{Orthogonal Circles as per our output}
    \label{fig:my_label}
\end{figure}
\section{Solution}
Standard form of a circle in matrix form is \\
\begin{center}
    $\vec{xVx^T} + 2\vec{u^Tx}$ + f = 0 \\
\end{center}
where, $$\vec{V} = \myvec{ 1 & 0 \\ 0 & 1}, Center \vec{C} = -\vec{V^{-1}u^T}$$\\
\\
Equation of given circles can be represented in matrix form as\\
\begin{equation}
    \vec{xx^T} + 2\myvec{ 1 & k}\vec{x} + 6 = 0 \label{eq-1}
\end{equation}
\begin{equation}
    \vec{xx^T} + 2\myvec{ 0 & k}\vec{x} + k = 0 \label{eq-2}
\end{equation}

where,\\
$$\vec{u_1} = \myvec{ 1 \\ k}, \vec{u_2} = \myvec{ 0 \\ k}, f_1 = 6 , f_2 = k$$
$$\vec{C_1} = \myvec{-1 & -k}, \vec{C_2} = \myvec{0 & -k}$$
Given that, the two circles are orthogonal so tangents at the point of intersection are orthogonal also at the point of intersection tangent of one circle passes through the center of other circle vice versa.\\
\\
Let $\vec{P}=\vec{X}$ be the point of intersection of two circles\\
\\
So, equation of tangents to the circle which passes through center of other circle can be written for both circles as\\
\\
\begin{equation}
    \myvec{-1 & -k}\vec{X} = 1 + k^2 \label{eq-3}
\end{equation}
\begin{equation}
    \myvec{0 & -k}\vec{X} = 1 + k^2 \label{eq-4}
\end{equation}
By solving eq-3, eq-4 we can obtain the point of intersection of two Circles as\\
\begin{center}
    $\vec{X}$ = $\myvec{0 \\ \frac{k^2 + 1}{k}}$
\end{center}

Also radius vectors of circles at the point of intersection are orthogonal so from figure 1, the angle between the radii r1 and r2 is 90\textdegree.  \\
\\
From figure \tridelt.PC1C2 is a Right angled triangle with hypotenuse d = $||C1-C2||$ .  \\
\\
So, by using \textbf{Pythagoraus theorem}  \\
\\
\begin{equation}
    \vec{||C_1 - C_2||^2 = ||C_1 - P||^2+||C_2 - P||^2}
\end{equation}
\\
Therefore,\\
\begin{equation}
    1 = 1+ \frac{2k^2 + 1}{k} + 0 + \frac{2k^2 + 1}{k}
\end{equation}
Yielding, k = $\frac{\vec{j}}{\sqrt{2}}$ or $\frac{-\vec{j}}{\sqrt{2}}$ \\
Therefore, the value of k is \\
\begin{center}
    \textbf{k = $\frac{\vec{j}}{\sqrt{2}}$ or $\frac{-\vec{j}}{\sqrt{2}}$}\\
\end{center}
\end{document}
