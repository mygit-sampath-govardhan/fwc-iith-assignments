
\documentclass[journal,12pt,twocolumn]{IEEEtran}
\usepackage{graphicx}
\graphicspath{{./figs/}}{}
\usepackage{amsmath,amssymb,amsfonts}
\usepackage{gensymb}
\usepackage{stackengine}
\usepackage{scalerel}
\newlength\triwidth
\newcommand\tridelt[1]{%
  \setlength\triwidth{\widthof{#1\ }}%
  \stackengine{-.1\triwidth}{#1\ }%
    {\scaleto{\Delta}{1\triwidth}}{O}{c}{F}{F}{L}%
}
\title{
Matrix-circle
}
\author{Kukunuri Sampath Govardhan}
\begin{document}
\maketitle
\tableofcontents
\begin{abstract}
This document shows how to find a variable k given that 2 circles intersect Orthogonally using python.
\end{abstract}
\section{Problem Statement}
If circles x$^2$+y$^2$+2x+2ky+6 =0,  x$^2$+y$^2$+2ky+k =0 \\
intersect orthogonally then find k.\\
\section{Construction}
\begin{figure}[h]
    \centering
\includegraphics[width=\columnwidth]{figs/assign5.png}
    \caption{Orthogonal Circles C1 and C2}
    \label{fig:my_label}
\end{figure}

\begin{table}[h]
    \centering
    \begin{tabular}{|c|c|c|}
       \hline
       \textbf{Symbol}&\textbf{Value}&\textbf{Description}  \\
       \hline
        C1 & $\begin{pmatrix}
  -1\\
  -k\\
 \end{pmatrix}$% 
 & Center of circle C1\\
        \hline
        C2 & $\begin{pmatrix}
  0\\
  -k\\
 \end{pmatrix}$% 
 & Center of circle C2\\
        \hline
        r1 & $\sqrt{k^{2}-5}$ & Radius of circle C1 \\
        \hline
        r2 & $\sqrt{k^{2}-k}$ & radius of circle C2 \\
        \hline
        $\theta$ & 90\textdegree & Given that C1 and C2 are Orthogonal\\
        \hline
        d & 1 & Distance between centers of the circles C1 and C2\\
        \hline
    \end{tabular}
    \caption{Parameters}
    \label{tab:my_label}
\end{table}


\section{Solution}
Equation of a circle is $(x-a)^2 + (y-b)^2 = r^2$ with center C = $\begin{pmatrix}
    a \\
    b \\
\end{pmatrix}$
and radius r.\\
So,\\
The First circle can also be written as \\
\begin{equation}
    (x + 1)^2 + (y + k)^2 = k^2 - 5 \\
    \label{eq-1}
\end{equation}
The Second circle can also be written as \\
\begin{equation}
    (x - 0)^2 + (y + k)^2 = k^2 - k \\
    \label{eq-2}
\end{equation}
Given that the two circles are orthogonal so tangents at the point of intersection are also orthogonal also radius vectors of circles at the point of intersection are also orthogonal so from figure 1, the angle between the radii r1 and r2 is 90\textdegree.  \\
\\
From figure \tridelt.PC1C2 is a Right angled triangle with hypotenuse d = $||C1-C2||$ .  \\
\\
So, by using \textbf{Pythagoraus theorem}  \\
\\
$d^{2} = r1^{2} + r2^{2}$ \\
\\
Therefore,\\
\\
1 = $k^{2}-5+k^{2}-k$\\
\\
i.e, $2k^{2}-k-6 = 0$ \\
\\
Thus, k = 2 or $\frac{-3}{2}$ \\
\\
Therefore, the value of k is \\
\begin{center}
    \textbf{k = 2 or $\frac{-3}{2}$}\\
\end{center}
\section{Software}
Download the following code using,
\begin{table}[h]
    \centering
    \begin{tabular}{|c|}
    \hline \\
         svn co   \\
         \\
   \hline
    \end{tabular}
\end{table}
\\
and execute the code by using command
\begin{center}
\textbf{Python3  Assignment5.py}\\
\end{center}

\section{Conclusion}
We found a variable k given that 2 circles intersect Orthogonally.\\
\end{document}
